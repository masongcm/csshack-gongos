\documentclass[xcolor=table]{beamer}
\usepackage{amsmath,latexsym}
\usepackage{epsfig}
\usepackage{amssymb,indentfirst,verbatim,afterpage,caption,graphicx,subfigure,float,rotating}
\usepackage{epstopdf,bbm,fixmath}
\usepackage{color}
\usepackage{adjustbox}
\usepackage [english]{babel}
\pdfmapfile{+sansmathaccent.map}
\usepackage [autostyle, english = american]{csquotes}
\setbeamertemplate{caption}[numbered]
\setbeamertemplate{theorems}[numbered]
\beamertemplatenavigationsymbolsempty
\MakeOuterQuote{"}
\usetheme[secheader]{Madrid}
\usecolortheme{beaver}
\newcommand{\backupbegin}{
   \newcounter{framenumberappendix}
   \setcounter{framenumberappendix}{\value{framenumber}}
}
\newcommand{\backupend}{
   \addtocounter{framenumberappendix}{-\value{framenumber}}
   \addtocounter{framenumber}{\value{framenumberappendix}} 
}
\addtobeamertemplate{navigation symbols}{}{%
    \usebeamerfont{footline}%
    \usebeamercolor[fg]{footline}%
    \hspace{1em}%
  %  \insertframenumber/\inserttotalframenumber
}


\title[GONGOs]{\textbf{King(s) of the GONGOs? A text analysis of reports from the UNHRC Universal Periodic Review }}
\author
{LOADS OF PEOPLE}
\date[19th April, 2018]{19th April, 2018}

\begin{document}

\begin{frame}
\titlepage
\centering{\large{LSE Computational Social Science Hackathon}}
\end{frame}

\begin{frame}
\frametitle{\textbf{What} are GONGOs?}
\begin{itemize}
\item Government-organized non-governmental organization (GONGO) 
\item An organization sponsored/formed by a govt with the purpose of supporting its own domestic/international political interests 
\end{itemize}
\end{frame}

\begin{frame}
\frametitle{\textbf{Why} are GONGOs problematic?}
\begin{itemize}
\item Presence of GONGOs can portray societies as open and democratic (perhaps despite being otherwise)
%GONGOs present a dangerous illusion: GONGOs often give the illusion of an open and robust civil society. This is in many ways part of their purpose. Under tight control and clear directives, GONGOs, utilizing their superior access to funds and resources, join regional and international forums, workshops, and consultations. In doing so, they stay on script and help to produce a masterful public relations fa�ade (as long as no one ask them too many probing questions about human rights or other controversial matters).

\item Consume resources in terms of govt financial support, and, in general, gathering funds from other sources (relative to other NGOs)
%GONGOs consume resources: In presenting themselves as civil society organizations or non-government entities and operating in generally favorable environments, GONGOs are able to gather disproportionate funds relative to their non-GONGO counterparts.
\item Crowd out NGOs and detracting from the "main issues"
%GONGOs act as gatekeepers: As GONGOs enter into the civil society arena, they tend to co-opt and monopolize these spaces. Their participation in civil society shrinks space available to independent and genuine groups. Exclusion of these other groups is not a coincidence, but carefully calculated and deliberate.
%GONGOs dilute the conversation: As GONGOs go about their work the consequences to discussions, processes and outcomes are actually quite serious. Bound by their master, GONGOs are not able to talk about critical issues that civil society needs to discuss. They even go so far as to present distorted or false information or to block vital conversations entirely. This manifests in regional forums, workshops, and even in civil society shadow reports to UN Treaty Bodies. Unfortunately, it is precisely the most crucial issues that need to be discussed which get mischaracterized or brushed aside because of GONGOs. Importantly, this pollution of civil society, like other forms on pollution, lacks neat and tidy borders. It drifts well beyond being the problem of a particular country, as civil society at the regional level is held hostage.
\end{itemize}
\begin{itemize}
\item For example: At the ACSC conference in 2015, in Malaysia, funding distribution from member countries meant no independent stakeholders attended 
%or the ACSC/APF conference held in 17-19 November 2015 in Malaysia, ongoing undemocratic national processes have resulted in the exclusion of independent groups from several countries, including Vietnam, Laos, and Brunei. Limited scholarship distribution through this process has meant GONGOs are being fully funded to attend the conference while others have no chance of joining.
\end{itemize}
\end{frame}
\begin{frame}
\frametitle{\textbf{Who} are the GONGOs?}

\end{frame}
\end{document}
