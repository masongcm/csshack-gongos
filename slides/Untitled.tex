\documentclass[xcolor=table]{beamer}
\usepackage{amsmath,latexsym}
\usepackage{epsfig}
\usepackage{amssymb,indentfirst,verbatim,afterpage,caption,graphicx,subfigure,float,rotating}
\usepackage{epstopdf,bbm,fixmath}
\usepackage{color}
\usepackage{adjustbox}
\usepackage [english]{babel}
\pdfmapfile{+sansmathaccent.map}
\usepackage [autostyle, english = american]{csquotes}
\setbeamertemplate{caption}[numbered]
\setbeamertemplate{theorems}[numbered]
\beamertemplatenavigationsymbolsempty
\MakeOuterQuote{"}
\usetheme[secheader]{Madrid}
\usecolortheme{beaver}
\newcommand{\backupbegin}{
   \newcounter{framenumberappendix}
   \setcounter{framenumberappendix}{\value{framenumber}}
}
\newcommand{\backupend}{
   \addtocounter{framenumberappendix}{-\value{framenumber}}
   \addtocounter{framenumber}{\value{framenumberappendix}} 
}
\addtobeamertemplate{navigation symbols}{}{%
    \usebeamerfont{footline}%
    \usebeamercolor[fg]{footline}%
    \hspace{1em}%
  %  \insertframenumber/\inserttotalframenumber
}


\title[CC \& RP]{\textbf{Cost of Consideration and Revealed Preference}}
\author
{Gavin Kader}
\institute{UCL}
\date[12th April, 2018]{12th April, 2018}

\begin{document}

\begin{frame}
\titlepage
\centering{\large{2018 ENTER Jamboree, TSE}}
\end{frame}

\section{Motivation}
\subsection{}
\begin{frame}
\frametitle{What is a consideration set?}
\begin{itemize}
\pause
\item  First introduced by Howard and Sheth (1969), Wright and Barbour (1977)
\item A subset of the total number of goods with which a consumer makes their decisions
\item For example:
	\begin{itemize}
	\item Over 20 tea companies in the UK selling multiple types of tea
	\item A consideration set (for a particular individual) could involve teas only sold by PG Tips and Tetley
	\end{itemize}
\pause
\item Set of goods used to make decisions \textit{"...need not coincide with the set of all possible alternatives"} (Horowitz et. al (1995))
\item Costly to use the set of total alternatives
\end{itemize}
\end{frame}

\end{document}
